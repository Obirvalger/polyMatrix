% \documentclass[a4paper, 14pt]{extarticle}
\documentclass[bibliography=totoc, a4paper, 14pt]{extarticle}
\usepackage{setspace}
\usepackage {indentfirst}
\setstretch{1.3}
\usepackage[utf8]{inputenc}
\usepackage[russian]{babel}
\usepackage{cite}
\usepackage{hyperref}
\usepackage{graphicx}
\usepackage[nottoc,notlot,notlof]{tocbibind}
\usepackage{icomma}
\let\stdsection\section
\renewcommand\section{\newpage\stdsection}
\addto\captionsrussian{% Replace "english" with the language you use
  \renewcommand{\contentsname}%
    {Оглавление}%
}
\hypersetup{
    colorlinks,
    allcolors=blue
}
\usepackage{longtable, moreverb}
\usepackage{ amssymb, latexsym, amsmath, amsthm}
\newtheorem{myth}{Теорема}
\newtheorem{mylm}{Лемма}
\newtheorem*{myco}{Следствие}
\usepackage{verbatim}
\newcommand{\rol} {\textrm{rol}}
\newcommand{\pphi}[1] {P^{\delta}(\varphi_{#1})}
\textwidth=17cm       % Эти строки нужны для того,
\textheight=21cm      % чтобы равенства уместились
\oddsidemargin=0cm    % на странице

\sloppy
% \fussy

\xdef\LastDeclaredEncoding{T2A}

\begin{document}
\setcounter{page}{3}
\setcounter{secnumdepth}{-1}

\tableofcontents

\section{Введение}
Одним из стандартных способов задания функций k\nobreakdash-значной логики являются поляризованные
полиномиальные формы (ППФ), которые также называются обобщенными формами Рида-Мюллера, или
каноническими поляризованными полиномами. В ППФ каждая переменная имеет определенную поляризацию.
Длиной полиномиальной формы называется число попарно различных слагаемых в ней. Длиной функции $f$
в классе ППФ называется наименьшая длина среди длин всех поляризованных полиномиальных форм,
реализующих $f$. Функция Шеннона $L^K_k(n)$ длины определяется как наибольшая длина среди всех
функций $k$\nobreakdash-значной логики в классе $K$ от~$n$~переменных, если $K$ опущено, то
подразумевается класс ППФ. Практическое применение ППФ нашли при построении программируемых
логических матриц (ПЛМ)~\cite{ue04, sb90}, сложность ПЛМ напрямую зависит от длины ППФ, по которой
она построена. Поэтому в ряде работ исследуется сложность ППФ различных функций
\cite{sv93,pn95,ss02,kk05,sd08,mn12,sm09}.

В 1993  В.\,П.\,Супрун~\cite{sv93} получил первые оценки функции Шеннона для функций алгебры логики :
$$
L_2(n) \geqslant C_n^{[\frac{n}{2}]},
$$
$$
L_2(n) < 3 \cdot 2^{n-1},
$$
где [$a$] обозначает целую часть $a$.

Точное значение функции Шеннона для функций алгебры логики в 1995\,г. было
найдено Н.\,А.\,Перязевым~\cite{pn95} :
$$
L_2(n) = \left[\frac{2^{n+1}}{3}\right].
$$

Функции $k$\nobreakdash-значных логик являются естественным обобщением функций алгебры логики.
Для функций $k$\nobreakdash-значной логики верхняя оценка функции Шеннона была получена в 2002\,г. С.\,Н.\,Селезневой~\cite{ss02} :
$$
L_k(n) < \frac{k(k-1)}{k(k-1)+1}k^n.
$$

При построении ПЛМ рассматривают и другие полиномиальные формы. Например класс обобщенных полиномиальных форм.
В классе обобщенных полиномиальных форм, в отличие от класса поляризованных полиномиальных форм, переменные могут иметь
различную поляризацию в разных слагаемых. В статье К.\,Д.\,Кириченко~\cite{kk05}, опубликованной в 2005\,г., получена верхняя оценка
функции Шеннона в классе обобщенных полиномиальных форм функций алгебры логики :
$$
L^{\text{О.П.}}_2(n) \leqslant \frac{2 ^ {n + 1}(\log_2n+1)}{n}.
$$

Верхняя оценка функции Шеннона в классе обобщенных полиномиальных форм функций k\nobreakdash-значной логики была получена
С.\,Н.\,Селезневой и А.\,Б.\,Дайняком в 2008\,г.~\cite{sd08}:
$$
L^{\text{О.П.}}_k(n) \lesssim 2\cdot\frac{k ^ n}{n}\cdot \ln n \text{ при } n \rightarrow \infty.
$$

В 2012\,г. Н.\,К.\,Маркеловым была получена нижняя оценка функции Шеннона для функции трехзначной логики в классе
поляризованных полиномов~\cite{mn12}:
$$
L_3(n) \geqslant \left[\frac{3}{4}3^n\right].
$$

\section{Основные определения}

Пусть $k \geqslant 2$ -- натуральное число, $E_k = \{0, 1, \dots, k - 1\}$. Весом набора
$\alpha = (a_1, \dots, a_n ) \in E_k^n$ назовем число $|\alpha| = \sum\limits_{i=1}^n a_i$.
Моном $\prod\limits_{a_i\neq0}x_i^{a_i}$ назовем соответствующим набору $\alpha =
(a_1, \dots, a_n ) \in E_k^n$ и обозначим через $K_{\alpha}$. По определению положим, что константа
1 соответствует набору из всех нулей. Функцией $k$\nobreakdash-значной логики называется
отображение $f^{(n)} : E_k^n \rightarrow E_k$, $n = 0, 1, \dots$.
Множество всех функций $k$-значной логики обозначим через $P_k$ , множество всех функций
$k$-значной логики, зависящих от переменных $x_1, \dots, x_n$ , обозначим через $P_k^n$.
Функция $j_i(x) = \begin{cases} 1, \text{ если } x = i; \\
                                0, \text{ если } x \neq i. \end{cases}$
% При простом $k$ $j_i(x)$ может быть представлена ввиде $j_i(x) = 1-(x-i)^{k-1}$.

Если $k$ -- простое число, то каждая функция $k$\nobreakdash-значной логики $f(x_1 , \dots , x_n)$
может быть однозначно задана формулой вида

$$ f(x_1, \dots, x_n) = \sum_{\alpha \in E_k^n:c_f(\alpha) \neq 0}c_f(\alpha)K_\alpha \; ,$$
где $c_f(\alpha) \in E_k$ -- коэффициенты, $\alpha \in E_k$, и операции сложения и умножения
рассматриваются по модулю $k$. Это представление функций $k$\nobreakdash-значной
логики называется ее полиномом по модулю $k$. При простых $k$ однозначно
определенный полином по модулю k для функции $k$\nobreakdash-значной логики $f$ будем
обозначать через $P(f)$.

Определим поляризованные полиномиальные формы по модулю $k$. Поляризованной переменной $x_i$ с поляризацией $d$,
$d \in E_k$ , назовем выражение вида $(x_i + d)$. Поляризованным мономом по вектору поляризации $\delta$,
$\delta = (d_1, \dots, d_n) \in E_k^n$, назовем произведение вида $(x_{i_1} + d_{i_1} )^{m_1}\cdots(x_{i_r} + d_{i_r})^{m_r}$,
где $1 \leqslant i_1 < \ldots < i_r \leqslant n$, и $1 \leqslant m_1 , \dots , m_r \leqslant k - 1$. Обычный моном является
мономом, поляризованным по вектору $\tilde{0} = (0, \dots, 0) \in E_k^n $.

Выражение вида $\sum\limits_{i=1}^lc_i \cdot K_i$, где $c_i \in E_k\setminus\{0\}$ -- коэффициенты, $K_i$ -- попарно
различные мономы, поляризованные по вектору $\delta = (d_1, \dots, d_n) \in E_k^n$, $i = 1, \dots , l$, назовем
поляризованной полиномиальной нормальной формой (ППФ) по вектору поляризации $\delta$. Мы будем считать, что константа 0
является ППФ по произвольному вектору поляризации. Заметим, что при простых $k$ для каждого вектора поляризации каждую функцию
$k$\nobreakdash-значной логики можно однозначно представить ППФ по этому вектору поляризации \cite{ss02}. При простых $k$
однозначно определенную ППФ по вектору поляризации $\delta \in E_k^n$ для функции
$f \in P_k^n$ будем обозначать через $P^{\delta}(f)$.

Длиной $l(p)$ ППФ $p$ назовем число попарно различных слагаемых в этой
ППФ. Положим, что $l(0) = 0$. При простых $k$ длиной функции $k$\nobreakdash-значной
логики в классе ППФ называется величина $l^{\text{ППФ}}(f) = \min\limits_{\delta \in E_k^n}l(P^{\delta}(f))$.

Функция $k$\nobreakdash-значной логики $f(x_1 ,\dots , x_n)$ называется симметрической, если
$$f(\pi(x_1), \dots, \pi(x_n)) = f(x_1, \dots, x_n)$$
для произвольной перестановки $\pi$ на множестве переменных $\{x_1 , \dots , x_n \}$.
Множество всех симметрических функций $k$\nobreakdash-значной логики обозначим через $S_k$.
Симметрическая функция $f(x_1, \dots, x_n)$ называется периодической c
периодом $\tau = (\tau_0 \tau_1 \dots \tau_{T-1}) \in E_k^T$ , если $f(\alpha) = \tau_j$ при $|\alpha| = j \pmod T$
для каждого набора $\alpha \in E_k^n$. При этом число $T$ называется длиной периода. Периодическую функцию
$k$\nobreakdash-значной логики $f(x_1 , \dots , x_n)$ с периодом $\tau = (\tau_0 \tau_1 \dots \tau_{T-1}) \in E_k^T$
будем обозначать через $f^{(n)}_{(\tau_0 \tau_1 \dots \tau_{T-1})}$. Понятно, что
такое обозначение полностью определяет эту функцию.

% Введем функцию $\rol(\alpha, i) \in E_k^n \times E_k \rightarrow E_k^n$, производящую чиклический сдвиг вектора $\alpha$
% влево. Пусть $\alpha = (a_1, \dots, a_n)$, тогда $\rol(\alpha, i) = (a_{(1+i)\mod k}, \dots, a_{(n+i)\mod k})$.

Пусть $T \geqslant  1$, $s \geqslant 1$, $\Pi = \{\tau_1, \dots, \tau_s | \tau_i \in E_k^T\}$,
$A_{\Pi} = \{f_{\tau}^{(n)}|\tau \in \Pi, n \geqslant 1\}$. Класс $A_{\Pi}$ называется вырожденным,
если для любого $\tau \in \Pi$ верно, что $l(f_{\tau}^{(n)}) = \bar{o}(k^n)$, при $n\rightarrow
\infty$.

\newpage
\section{Постановка задачи}

\newpage
\section{Результаты}

\newpage
\section{Заключение}

\makeatletter
\renewcommand*{\@biblabel}[1]{\hfill#1.}
\makeatother

\begin{singlespace}
\begin{thebibliography}{0}
\bibitem{ue04} Угрюмов~Е.\,П. Цифровая схемотехника. СПб.: БХВ-Петербург, 2004.
\bibitem{sb90} Sasao T., Besslich P. On the complexity of mod-2 sum PLA’s  // IEEE Trans.on Comput. 39. N 2. 1990. P.~262--266.
\bibitem{sv93} Супрун~В.\,П. Сложность булевых функций в классе канонических поляризованных полиномов // Дискретная математика. 5.
    \textnumero 2. 1993. С. 111--115.
\bibitem{pn95} Перязев Н.\,А. Сложность булевых функций в классе полиномиальных поляризованных~форм // Алгебра и логика. 34.
    \textnumero 3. 1995. С. 323--326.
\bibitem{ss02} Селезнева С.\,H. О сложности представления функций многозначных логик поляризованными полиномами. Дискретная
    математика. 14. \textnumero 2. 2002. С.~48--53.
\bibitem{kk05} Кириченко~К.\,Д. Верхняя оценка сложности полиномиальных нормальных форм булевых функций
    // Дискретная математика. 17. \textnumero 3. 2005. С. 80--88.
\bibitem{sd08} Селезнева С.\,Н. Дайняк А.\,Б. О сложности обобщенных полиномов k\nobreakdash-значных функций // Вестник Московского
    университета. Серия 15. Вычислительная математика и кибернетика. \textnumero 3. 2008. С. 34--39.
\bibitem{mn12} Маркелов Н.\,К. Нижняя оценка сложности функций трехзначной логики в классе поляризованных полиномов // Вестник
    Московского университета. Серия 15. Вычислительная математика и кибернетика. \textnumero 3. 2012. С. 40--45.
\bibitem{sm09} Селезнева С.\,H. Маркелов Н.\,К. Быстрый алгоритм построения векторов коэффициэнтов поляризованных полиномов
    k-значных функций // Ученые записки Казанского университета. Серия Физико-математические науки. 2009. 151.
    \textnumero 2 С.~147-151.
\end{thebibliography}

\end{singlespace}

\end{document}
